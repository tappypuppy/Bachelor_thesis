%!TEX root = ../main.tex
%\documentclass[twoside,twocolumn,a4j,dvipdfmx]{jarticle}
%\usepackage{amsmath,amssymb}
%\begin{document}
数値計算は様々な方程式、特に解析的に解くことが困難な方程式を計算機を用いて数値的に解く技術である。この技術の発展によって、気象データを元に今後の天気を予測すること、自動車や飛行機、船などの周囲の流体の流れを、実物を作らずともコンピュータ上で計算し様々なシミュレーションをすることなどが可能になった。

しかし、現代社会を支える数値計算は有限桁の浮動小数点数を利用して問題を解くため、有限桁で表現できない部分で誤差が生まれる。単純な数値計算であれば誤差はとても小さく憂慮するべきものではないが、大規模な数値計算になればなるほど誤差は大きくなり計算の信頼性に関わる。最終的には、数値計算で作られたシステムの安全性に影響し、最悪の場合誤差によって予期せぬ事故に繋がる可能性もある。
このような数値計算のリスクを制御するために、数値計算において生じる誤差を厳密に評価することで数学的に厳密な結果を導く手法を精度保証付き数値計算と言う\cite{seidohoshou}。

精度保証付き数値計算は現在、MATLABという計算機言語でS.M. Rumpによって開発された区間演算ライブラリINTLABを用いて実現できる。MATLABは数値計算の開発分野において著名なソフトウェアである。しかし、MATLAB、INTLABはともに有料のソフトウェアであり、精度保証付き数値計算の導入の敷居を高くしている。

この解決策として、オープンソースな計算機言語であるJuliaに着目した。JuliaはJeff Bezansonらによって開発され、2012年にオープンソースの理念のもと公開された新しい計算機言語であり、シンプルな文法や高速な実行速度を特徴に持つ。精度保証付き数値計算の開発が、オープンソース上で行われるようになれば、導入の敷居が低くなり新規参入者も増え、より一層の発展が見込める。

そこで、本研究では、Juliaを用いて常微分方程式の周期解の精度保証を実装し、一例として、van der Pol方程式の周期解の精度保証を行うことを目的とする。具体的には周期解をフーリエ級数で表現\cite{FourierSpectol} し、フーリエ係数に対する零点探索問題を考えることでNewton-Kantorovich型定理の成立を数値検証\cite{radiipolynomial,JPLessard}する。実装には高速フーリエ変換の区間演算\cite{FFT}の実装などが必須となる。
%\end{document}