%!TEX root = ../main.tex
%\documentclass[twoside,twocolumn,a4j,dvipdfmx]{jarticle}
%\usepackage{amsmath,amssymb}
%\usepackage[dvipdfmx]{graphicx}
%\newcommand{\im}{\mathrm{i}}
%\newcommand{\bx}{\mathrm x}
%\newcommand{\R}{\mathbb{R}}
%\newcommand{\Largezero}{\mbox{\Large{0}}}
%\newcommand{\Hugezero}{\mbox{\Huge{0}}}
%\newtheorem{thm}{定理}[section]
%\newtheorem{df}[thm]{定義}
%\newtheorem{lem}{補助定理}[section]
%\newtheorem{prop}[thm]{補題}
%\setcounter{MaxMatrixCols}{100}

%\begin{document}
%\section{おわりに}
本研究では、van der Pol方程式の周期解をフーリエ級数で表現し、フーリエ係数に対する零点探索問題を考えることで、Newton Kantorovich型定理の成立を数値検証した。今後の展望は、JuliaとMATLAB上の区間演算パッケージINTLABとの計算コストの比較を行い、Juliaの有用性を検証したい。また、他の方程式についてもJuliaで実装していきたい。
%\end{document}